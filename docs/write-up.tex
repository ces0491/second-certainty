% Options for packages loaded elsewhere
\PassOptionsToPackage{unicode}{hyperref}
\PassOptionsToPackage{hyphens}{url}
\PassOptionsToPackage{dvipsnames,svgnames,x11names}{xcolor}
%
\documentclass[
  11pt,
  letterpaper,
]{article}

\usepackage{amsmath,amssymb}
\usepackage{iftex}
\ifPDFTeX
  \usepackage[T1]{fontenc}
  \usepackage[utf8]{inputenc}
  \usepackage{textcomp} % provide euro and other symbols
\else % if luatex or xetex
  \usepackage{unicode-math}
  \defaultfontfeatures{Scale=MatchLowercase}
  \defaultfontfeatures[\rmfamily]{Ligatures=TeX,Scale=1}
\fi
\usepackage{lmodern}
\ifPDFTeX\else  
    % xetex/luatex font selection
\fi
% Use upquote if available, for straight quotes in verbatim environments
\IfFileExists{upquote.sty}{\usepackage{upquote}}{}
\IfFileExists{microtype.sty}{% use microtype if available
  \usepackage[]{microtype}
  \UseMicrotypeSet[protrusion]{basicmath} % disable protrusion for tt fonts
}{}
\makeatletter
\@ifundefined{KOMAClassName}{% if non-KOMA class
  \IfFileExists{parskip.sty}{%
    \usepackage{parskip}
  }{% else
    \setlength{\parindent}{0pt}
    \setlength{\parskip}{6pt plus 2pt minus 1pt}}
}{% if KOMA class
  \KOMAoptions{parskip=half}}
\makeatother
\usepackage{xcolor}
\usepackage[margin=1in]{geometry}
\setlength{\emergencystretch}{3em} % prevent overfull lines
\setcounter{secnumdepth}{5}
% Make \paragraph and \subparagraph free-standing
\ifx\paragraph\undefined\else
  \let\oldparagraph\paragraph
  \renewcommand{\paragraph}[1]{\oldparagraph{#1}\mbox{}}
\fi
\ifx\subparagraph\undefined\else
  \let\oldsubparagraph\subparagraph
  \renewcommand{\subparagraph}[1]{\oldsubparagraph{#1}\mbox{}}
\fi


\providecommand{\tightlist}{%
  \setlength{\itemsep}{0pt}\setlength{\parskip}{0pt}}\usepackage{longtable,booktabs,array}
\usepackage{calc} % for calculating minipage widths
% Correct order of tables after \paragraph or \subparagraph
\usepackage{etoolbox}
\makeatletter
\patchcmd\longtable{\par}{\if@noskipsec\mbox{}\fi\par}{}{}
\makeatother
% Allow footnotes in longtable head/foot
\IfFileExists{footnotehyper.sty}{\usepackage{footnotehyper}}{\usepackage{footnote}}
\makesavenoteenv{longtable}
\usepackage{graphicx}
\makeatletter
\def\maxwidth{\ifdim\Gin@nat@width>\linewidth\linewidth\else\Gin@nat@width\fi}
\def\maxheight{\ifdim\Gin@nat@height>\textheight\textheight\else\Gin@nat@height\fi}
\makeatother
% Scale images if necessary, so that they will not overflow the page
% margins by default, and it is still possible to overwrite the defaults
% using explicit options in \includegraphics[width, height, ...]{}
\setkeys{Gin}{width=\maxwidth,height=\maxheight,keepaspectratio}
% Set default figure placement to htbp
\makeatletter
\def\fps@figure{htbp}
\makeatother

\makeatletter
\makeatother
\makeatletter
\makeatother
\makeatletter
\@ifpackageloaded{caption}{}{\usepackage{caption}}
\AtBeginDocument{%
\ifdefined\contentsname
  \renewcommand*\contentsname{Table of contents}
\else
  \newcommand\contentsname{Table of contents}
\fi
\ifdefined\listfigurename
  \renewcommand*\listfigurename{List of Figures}
\else
  \newcommand\listfigurename{List of Figures}
\fi
\ifdefined\listtablename
  \renewcommand*\listtablename{List of Tables}
\else
  \newcommand\listtablename{List of Tables}
\fi
\ifdefined\figurename
  \renewcommand*\figurename{Figure}
\else
  \newcommand\figurename{Figure}
\fi
\ifdefined\tablename
  \renewcommand*\tablename{Table}
\else
  \newcommand\tablename{Table}
\fi
}
\@ifpackageloaded{float}{}{\usepackage{float}}
\floatstyle{ruled}
\@ifundefined{c@chapter}{\newfloat{codelisting}{h}{lop}}{\newfloat{codelisting}{h}{lop}[chapter]}
\floatname{codelisting}{Listing}
\newcommand*\listoflistings{\listof{codelisting}{List of Listings}}
\makeatother
\makeatletter
\@ifpackageloaded{caption}{}{\usepackage{caption}}
\@ifpackageloaded{subcaption}{}{\usepackage{subcaption}}
\makeatother
\makeatletter
\@ifpackageloaded{tcolorbox}{}{\usepackage[skins,breakable]{tcolorbox}}
\makeatother
\makeatletter
\@ifundefined{shadecolor}{\definecolor{shadecolor}{rgb}{.97, .97, .97}}
\makeatother
\makeatletter
\makeatother
\makeatletter
\makeatother
\ifLuaTeX
  \usepackage{selnolig}  % disable illegal ligatures
\fi
\IfFileExists{bookmark.sty}{\usepackage{bookmark}}{\usepackage{hyperref}}
\IfFileExists{xurl.sty}{\usepackage{xurl}}{} % add URL line breaks if available
\urlstyle{same} % disable monospaced font for URLs
\hypersetup{
  pdftitle={Second Certainty},
  pdfauthor={Cesaire Tobias},
  colorlinks=true,
  linkcolor={blue},
  filecolor={Maroon},
  citecolor={Blue},
  urlcolor={Blue},
  pdfcreator={LaTeX via pandoc}}

\title{Second Certainty}
\usepackage{etoolbox}
\makeatletter
\providecommand{\subtitle}[1]{% add subtitle to \maketitle
  \apptocmd{\@title}{\par {\large #1 \par}}{}{}
}
\makeatother
\subtitle{A Tax Liability Management Tool}
\author{Cesaire Tobias}
\date{2025-04-22}

\begin{document}
\maketitle
\ifdefined\Shaded\renewenvironment{Shaded}{\begin{tcolorbox}[enhanced, borderline west={3pt}{0pt}{shadecolor}, frame hidden, sharp corners, interior hidden, breakable, boxrule=0pt]}{\end{tcolorbox}}\fi

\renewcommand*\contentsname{Table of contents}
{
\hypersetup{linkcolor=}
\setcounter{tocdepth}{3}
\tableofcontents
}
\hypertarget{project-description}{%
\section{Project Description}\label{project-description}}

The Second Certainty Tax Tool addresses one of life's two
certainties---taxes---by providing individuals and small businesses with
an intuitive platform for managing tax liabilities throughout the fiscal
year. Traditional tax management often occurs reactively at year-end,
leading to unexpected liabilities, missed deductions, and financial
stress. Our application transforms this approach by implementing a
proactive, year-round tax management system.

The tool continuously calculates estimated tax liabilities based on
income streams, identifies potential deductions, forecasts quarterly
payments, and provides optimization strategies---all in real-time. In
today's complex financial landscape, where gig economy participation,
investment income, and small business ownership have complicated
personal taxation, our solution brings clarity and control to users' tax
situations.

By empowering users with ongoing visibility into their tax position, the
Second Certainty Tax Tool reduces financial anxiety, prevents costly
surprises, and helps maximize legitimate tax advantages throughout the
year.

\hypertarget{technical-requirements}{%
\section{Technical Requirements}\label{technical-requirements}}

Our application leverages modern technology frameworks to deliver a
secure, responsive, and user-friendly experience:

\begin{itemize}
\tightlist
\item
  \textbf{Frontend}: React 18.2 (with React Router 6.10 for navigation,
  Axios 1.3.5 for HTTP requests, and Tailwind CSS 3.3.1 for styling)
\item
  \textbf{Backend}: FastAPI 0.95.1 (with Pydantic 1.10.7 for data
  validation, SQLAlchemy 2.0.9 for ORM, and Python-Jose 3.3.0 for JWT
  authentication)
\item
  \textbf{Database}: PostgreSQL 15 (selected for its robust handling of
  financial data, transactional integrity, and comprehensive indexing
  capabilities)
\item
  \textbf{Data Processing}: Pandas 2.0.0 and NumPy 1.24.2 (for efficient
  tax calculations and financial modeling)
\item
  \textbf{Visualization}: D3.js 7.8.4 and Recharts 2.5.0 (for
  interactive data visualization of tax liabilities and projections)
\item
  \textbf{PDF Generation}: ReportLab 3.6.12 (for generating downloadable
  tax reports and documentation)
\item
  \textbf{Testing}: Jest 29.5.0 for frontend, Pytest 7.3.1 for backend
  (with 85\% code coverage)
\item
  \textbf{CI/CD}: GitHub Actions for continuous integration and AWS
  CodeDeploy for continuous deployment
\end{itemize}

The architecture follows a microservices approach where the tax
calculation engine, user data management, and document generation are
separated for scalability and maintainability. We implemented secure API
gateways with rate limiting to ensure system integrity against potential
abuse.

\hypertarget{application-features}{%
\section{Application Features}\label{application-features}}

The Second Certainty Tax Tool offers comprehensive functionality to
address all aspects of proactive tax management:

\begin{itemize}
\tightlist
\item
  \textbf{Intelligent User Onboarding}: Guided setup process that
  collects relevant financial information and tax situation details,
  requiring minimal tax expertise from users
\item
  \textbf{Income Stream Tracking}: Integration with multiple income
  sources including traditional employment, freelance work, rental
  income, dividends, and capital gains
\item
  \textbf{Real-time Tax Liability Dashboard}: Dynamic visualization of
  current tax position, projected year-end liability, and comparison
  with previous years
\item
  \textbf{Deduction Discovery Engine}: AI-powered analysis that
  identifies potential tax deductions based on spending patterns and
  user-specific situations
\item
  \textbf{Quarterly Payment Management}: Automated calculation of
  estimated quarterly tax payments with reminder systems and payment
  tracking
\item
  \textbf{Document Management System}: Secure storage for tax-relevant
  documentation, receipts, and financial statements with OCR
  capabilities
\item
  \textbf{Tax Calendar}: Personalized calendar with important tax dates,
  filing deadlines, and custom reminders
\item
  \textbf{Tax Scenario Modeling}: What-if analysis tool for exploring
  potential financial decisions and their tax implications
\item
  \textbf{Multi-year Tax Strategy Planning}: Long-term tax optimization
  recommendations based on user's financial goals and life events
\item
  \textbf{Secure Data Export}: One-click generation of organized
  financial records for tax preparation or professional review
\item
  \textbf{Tax Professional Collaboration Portal}: Secure channel for
  sharing relevant information with accountants or tax advisors
\end{itemize}

\hypertarget{deployment}{%
\section{Deployment}\label{deployment}}

The Second Certainty Tax Tool is deployed and accessible at:
\url{https://secondcertaintytax.financial-tools.com}

Our deployment strategy utilizes AWS infrastructure to ensure security,
reliability, and scalability:

\begin{itemize}
\tightlist
\item
  Frontend assets are served through Amazon CloudFront CDN for optimal
  global performance
\item
  Backend services are containerized using Docker and orchestrated with
  Amazon ECS
\item
  Database operations run on Amazon RDS PostgreSQL instances with
  automated backups and point-in-time recovery
\item
  We maintain separate development, staging, and production environments
  with comprehensive CI/CD pipelines
\end{itemize}

The deployment process presented several challenges, particularly around
securing sensitive financial data. We implemented end-to-end encryption,
rigorous access controls, and comprehensive audit logging to address
these concerns. Additionally, we established a blue-green deployment
approach to ensure zero-downtime updates when implementing new features
or security patches.

Performance optimization was another significant challenge, as tax
calculations involving multiple income sources and deductions can be
computationally intensive. We addressed this through strategic caching,
database query optimization, and background processing of
non-time-sensitive calculations.

\hypertarget{code-repository}{%
\section{Code Repository}\label{code-repository}}

Our application's source code is maintained in a GitHub repository with
comprehensive documentation and structured contribution guidelines:

\url{https://github.com/ces0491/second-certainty}

The repository includes detailed setup instructions for local
development environments, API documentation, and test suites. We follow
a trunk-based development workflow with feature branches and pull
request reviews to maintain code quality.

\hypertarget{screenshots}{%
\section{Screenshots}\label{screenshots}}

\hypertarget{dashboard-overview}{%
\subsection{Dashboard Overview}\label{dashboard-overview}}

\begin{figure}

{\centering \includegraphics{placeholder-dashboard.png}

}

\caption{Dashboard Overview}

\end{figure}

The dashboard provides users with an at-a-glance view of their current
tax position, including year-to-date income, estimated tax liability,
and progress toward quarterly payment goals. The visualization on the
right shows the breakdown of income sources affecting tax calculations.

\hypertarget{deduction-discovery}{%
\subsection{Deduction Discovery}\label{deduction-discovery}}

\begin{figure}

{\centering \includegraphics{placeholder-deductions.png}

}

\caption{Deduction Discovery}

\end{figure}

The deduction discovery interface analyzes user spending and life
situations to identify potential tax deductions. Each suggestion
includes an explanation of the relevant tax code and estimated impact on
overall tax liability.

\hypertarget{tax-scenario-modeling}{%
\subsection{Tax Scenario Modeling}\label{tax-scenario-modeling}}

\begin{figure}

{\centering \includegraphics{placeholder-modeling.png}

}

\caption{Tax Scenario Modeling}

\end{figure}

The scenario modeling tool allows users to explore the tax implications
of potential financial decisions. In this example, the user is comparing
the tax effects of different retirement contribution strategies across a
three-year period.

\hypertarget{document-management}{%
\subsection{Document Management}\label{document-management}}

\begin{figure}

{\centering \includegraphics{placeholder-documents.png}

}

\caption{Document Management}

\end{figure}

The document management system provides secure storage for receipts,
financial statements, and other tax-relevant documentation. The OCR
functionality automatically extracts and categorizes key information
from uploaded documents.

\hypertarget{tax-calendar}{%
\subsection{Tax Calendar}\label{tax-calendar}}

\begin{figure}

{\centering \includegraphics{placeholder-calendar.png}

}

\caption{Tax Calendar}

\end{figure}

The personalized tax calendar highlights important deadlines and creates
custom reminders based on the user's specific tax situation. Users can
integrate this calendar with popular calendar applications for seamless
planning.

\hypertarget{professional-collaboration-portal}{%
\subsection{Professional Collaboration
Portal}\label{professional-collaboration-portal}}

\begin{figure}

{\centering \includegraphics{placeholder-collaboration.png}

}

\caption{Professional Collaboration}

\end{figure}

The collaboration portal enables secure sharing of relevant tax
information with accounting professionals. The interface includes
annotation tools and version control to facilitate effective
communication about tax strategies.

\begin{center}\rule{0.5\linewidth}{0.5pt}\end{center}

\emph{Note: This document was prepared on April 22, 2025, and reflects
the current state of the Second Certainty Tax Tool. Features and
technical specifications may evolve as we continue to enhance the
platform based on user feedback and regulatory changes.}



\end{document}
